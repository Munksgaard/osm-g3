\documentclass [10pt,a4paper]{article}
\usepackage[danish]{babel}
\usepackage{a4wide}
\usepackage[T1]{fontenc}
\usepackage[utf8x]{inputenc}
\usepackage{amsmath}
\usepackage{amsfonts}
\usepackage{fancyhdr}
\usepackage{ucs}
\usepackage{graphicx}

\pagestyle{fancy}
\fancyhead[LO]{Daniel Egeberg \& Philip Munksgaard}
\fancyhead[RO]{OSM: G3}
\fancyfoot[CO]{\thepage}


\title{G3}
\author{Daniel Egeberg \& Philip Munksgaard}
\date{8. marts 2011}

\begin{document}
\maketitle

\section*{Opgave 1} % {{{

Vores struktur \verb+lock_t+ har en spinlock til at sikre atomisk adgang og et flag der afgør om den er låst eller ej.

Når \verb+lock_acquire+ bliver kaldt, kontrolleres først om låsen er fri. Hvis ikke lægges tråden, ved hjælp af en \verb+sleepq+, til at sove indtil der bliver kaldt \verb+lock_release+.

Det eneste \verb+lock_reset+ gør er at kalde \verb+sleepq_wake_all+, \verb+spinlock_reset+ og sætte \verb+is_locked+-flaget til nul. Da ingen af disse operationer kan fejle giver det ikke mening at returnere andet en nul.

% }}}

\section*{Opgave 2} % {{{

Vores \verb+cond_t+ er bare en peger til et heltal, som bruges som ressource til en \verb+sleepq+. 

Når en process kalder \verb+condition_wait+ bliver den tilføjet til sovekøen, condition låsen frigives og der foretages trådskifte.

\verb+condition_signal+ og \verb+condition_broadcast+ vækker bare henholdsvis en og alle processer i køen.
% }}}

\section*{Opgave 3} % {{{

Vi har sammenflettet vores kode med koden fra den vejledende løsning
til G2, som også indeholder \verb+fork+ systemkaldet.

% }}}

\section*{Opgave 4} % {{{

Vi har tilføjet de forskellige systemkald til \verb+proc/syscall.h+, \verb+proc/syscall.c+ samt \verb+test/lib.c+ og \verb+test/lib.h+.

Vi har valgt at mappe direkte fra \verb+usr_lock_t+ og \verb+usr_cond_t+ til henholdsvis \verb+lock_t+ og \verb+cond_t+.

% }}}

\end{document}
